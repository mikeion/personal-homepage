\documentclass[margin,line]{res}

\usepackage{pifont}
\usepackage{hyperref}
\usepackage[latin1] {inputenc}
\usepackage{fancyhdr}
\usepackage{amssymb,amsfonts}
\usepackage[utf8]{inputenc}
\usepackage{multirow}
\usepackage{longtable}
\usepackage{url}
\usepackage{enumitem}

\oddsidemargin -.15in
\evensidemargin -.15in
\textheight=9.0in

\pagestyle{fancy}
\renewcommand{\headrulewidth}{0pt}
\fancyhf{}
\newenvironment{list1}{
  \begin{list}{\ding{113}}{%
      \setlength{\itemsep}{0in}
      \setlength{\parsep}{0in} \setlength{\parskip}{0in}
      \setlength{\topsep}{0in} \setlength{\partopsep}{0in}
      \setlength{\leftmargin}{0.17in}}}{\end{list}}
\newenvironment{list2}{
  \begin{list}{$\bullet$}{%
      \setlength{\itemsep}{0in}
      \setlength{\parsep}{0in} \setlength{\parskip}{0in}
      \setlength{\topsep}{0in} \setlength{\partopsep}{0in}
      \setlength{\leftmargin}{0.2in}}}{\end{list}}
      \newenvironment{list3}{
        \begin{list}{\ding{113}}{%
            \setlength{\itemsep}{0in}
            \setlength{\parsep}{0in} \setlength{\parskip}{0in}
            \setlength{\topsep}{0in} \setlength{\partopsep}{0in}
            \setlength{\leftmargin}{0in}
            \setlength{\rightmargin}{1.25in}}}{\end{list}}

      \renewcommand{\subsection}[1]{%
      \par\vspace{3pt}% Adjust this space as necessary
      \underline{\normalsize\bfseries #1}%
      \par\vspace{3pt}% Adjust this space as necessary
    }
\begin{document}

\name{Michael (Mike) Ion \vspace*{.1in}}

\begin{resume}

\section{\sc Contact Information}

\begin{tabular}{@{}p{2in}p{4in}}
Marsal School of Education  & \href{mailto:mikeion@umich.edu}{Email: mikeion@umich.edu}  \\
University of Michigan & \href{https://github.com/mikeion}{GitHub: https://github.com/mikeion} \\
Ann Arbor, MI 48104  & \href{https://mikeion.com}{Webpage: https://mikeion.com} \\
\end{tabular}

\section{\sc Education}

{\bf University of Michigan}\\
\vspace*{-.1in}
\begin{list3}
\item[] Ph.D. in Mathematics Education, May 2024
\item[] Advisor: \href{https://deborahloewenbergball.com/}{Deborah Ball}, Commitee: (Cognate) \href{https://jurgens.people.si.umich.edu/}{David Jurgens}, \href{https://marsal.umich.edu/directory/faculty-staff/christopher-quintana}{Christopher Quintana}, \href{https://marsal.umich.edu/directory/faculty-staff/ying-xu}{Ying Xu}
\end{list3}

{\bf California Polytechnic State University}\\
\vspace*{-.1in}
\begin{list3}
\item[] M.S.~in Mathematics, June 2015
\item[] B.S.~in Mathematics, June 2013
\end{list3}

\section{\sc Research Interests}
\begin{list3}
\item Applications of Data Science and Machine Learning in \\
Educational Contexts
\item Quantitative Survey Methodologies for Educational Assessment
\item Large Language Models as a Tool for Educational Research
\item STEM Education Research

\end{list3}

\section{\sc Research Experience} 
{\bf University of Michigan}

\subsection{\sc Postdoctoral Research Fellow}
\begin{tabular}{@{}p{4.0in}p{1.0in}}
School of Information, University of Michigan (hybrid) & \multirow{2}{1.0in}{May~2024-Present}\\
Supervisor: Prof. Kevyn Collins-Thompson \\
I serve as a senior member of the Collins-Thompson lab, focusing on developing and disseminating creative, high-impact research contributions at the intersection of AI and education. My responsibilities include:
\begin{list1}
\item Conducting theoretical modeling, experimental design and implementation, and data analysis in AI and education research.
\item Publishing scholarly papers and presenting research at national and international conferences and meetings.
\item Writing grant proposals, either leading them or making significant contributions.
\item Mentoring undergraduate and graduate students working in the lab.
\item Developing research in areas such as teaching models for conversational AI, AI-based simulated student models, AI-assisted optimal practice scheduling, and AI-assisted methods for analyzing learning interaction data.
\item Creating new datasets and tasks, including gold-standard benchmarks.
\end{list1}
\end{tabular}
\newpage
\subsection{\sc Research Assistant}
\begin{tabular}{@{}p{4.0in}p{1.0in}}
Grasping Rationality and Instructional Practices (GRIP) Lab  & \multirow{2}{1.0in}{Sept.~2017-May~2023}\\
Supervisors: Pat Herbst and Amanda Brown (Milewski) \\
I served as the lead graduate student researcher for the Geometry for Teachers (GeT) Support Project, a \$2.3 million NSF IUSE Grant (\#1725837). My responsibilities included:
\begin{list1}
\item Conducting item-response theory (IRT) analysis of results from mathematical knowledge for teaching (MKT) assessments taken by university students, and distributing reports to their instructors.
\item Coding qualitative data (e.g., interview data, survey responses) manually and with machine learning models.
\item Organizing working groups for an online professional development network of university geometry instructors from across the US.
\item Developing and analyzing psychometric survey instruments to be taken to assess and understand the nature of the university geometry course.
\item Writing conference papers and presenting research at national conferences.
\end{list1}
\end{tabular}
\vspace{0.3cm}
\begin{tabular}{@{}p{4.0in}p{1.0in}}
  College and Beyond II Project (Mellon Grant)  & \multirow{2}{1in}{Apr.~2020-Jun~2022}\\
Supervisor: Anne Gere \\
I ran statistical analyses and provided insight to a research team studying the effects of a liberal arts education on life outcomes. Additionally, our team reported our findings at two conferences and published one journal article from our work. My responsibilities included:
\begin{list1}
  \item Working on a team to analyze results from a pilot survey to determine next steps forward.
  \item Providing readability statistics on a set of essay responses.
  \item Connecting responses on a pilot survey using structural equation modeling.
\end{list1}
\end{tabular}

\begin{tabular}{@{}p{4.0in}p{1.0in}}
  Wolverine Pathways Curriculum Development Project  & \multirow{2}{1in}{May~2019-Dec~2019}\\
Supervisor: Maisie Gholson \\
I developed social-justice oriented mathematics curriculum materials for a \href{https://lsa.umich.edu/csp/bridge-programs/sbsp-courses.html}{summer bridge program}. I helped facilitate a professional development workshop for the teachers of the summer program and advised team members on  quantitative survey methodologies. 
\end{tabular}

\newpage
{\bf Journal for Research in Mathematics Education (JRME)}

\subsection{\sc Editorial Assistant}
\begin{tabular}{@{}p{4in}p{1.0in}}
 National Council of Teachers of Mathematics (NCTM)  & \multirow{2}{1.0in}{Jan~2022-Mar~2023}\\
As an Editorial Assistant on the Editorial Board of JRME, my primary responsibility was to review manuscripts being considered for publication. My focus was on ensuring the accuracy and validity of statistical methods and advanced mathematics used in the text and accompanying figures/tables. I meticulously checked for typos and errors while also reviewing the grammar and writing style to guarantee clear and effective communication of the findings and statistics presented in the manuscript.
\end{tabular}

{\bf California Polytechnic State University}
\subsection{\sc Research Experience for Undergraduates (REU)}
\begin{tabular}{@{}p{4in}p{0.4in}p{0.8in}}
  Research Topic: Stanley's Conjecture, Cover Depth, and Simplicial  Complexes & \multirow{2}{1.0in}{Jun~2013-Sept~2013}\\
  Research Advisor: Ben Richert \\
\end{tabular}\\


\section{\sc Publications}

\subsection{\sc Peer-Reviewed Journal Articles} 
\begin{list3} 
  \item \textbf{Ion, M.}, Herbst, P. (In review). Measuring Tacit Mathematics Teaching Knowledge: A Natural Language Processing Approach. \textit{Journal of the Learning Sciences}. 
  \item Paulsen, A., Godfrey, J., \textbf{Ion, M.}, (In review). Writing Across the Curriculum: a Text as Data Approach. \textit{Educational Effectiveness and Policy Analysis}. 
  \item Short, C., \textbf{Ion, M.} (In progress). Generative Artificial Intelligence for Theory Building. \textit{Academy of Management Review}. 
  \item Herbst, P., Brown, A.M., \textbf{Ion, M.}, Margolis, C. (2023). Teaching Geometry for Secondary Teachers: What are the Tensions Instructors Need to Manage? \textit{International Journal of Research in Undergraduate Mathematics Education}. (2023). \url{https://doi-org.proxy.lib.umich.edu/10.1007/s40753-023-00216-0}
  \item Gere, A., Godfrey, J., Griffin, M., \textbf{Ion, M.}, Limlamai, N., Moos, A., Van Zanen, K. (2023). Alumni Perspectives on General Education: How Writing Can Increase What We Know. \textit{Journal of General Education, 70}(1-2), 149-175. \url{https://doi.org/10.5325/jgeneeduc.70.1-2.0149}
\end{list3}

\subsection{\sc Peer-Reviewed Conference Proceedings} 
\begin{list3} 
  \item \textbf{Ion, M.}, Herbst, P., Ko, I., Hetrick, C. (Oct. 2023). Surveying Instructors of Geometry for Teachers Courses: An Illustration of Balanced Incomplete Block Design. \textit{Psychology of Mathematics Education, North America Annual Conference}. Reno, NV. 
  \item Brown, A., Herbst, P., \textbf{Ion, M.} (Oct. 2023). How Instructors of Undergraduate Mathematics Courses Manage Tensions Related to Teaching Courses for Teachers. \textit{Psychology of Mathematics Education, North America Annual Conference}. Reno, NV. 
  \item Boyce, S., An, T., Pyzdrowski, L., Oppong-Wadie, K., \textbf{Ion, M.}, St. Goar, J. (Feb. 2023). Learning from Lesson Study in the College Geometry Classroom. \textit{25th Annual Conference on Research in Undergraduate Mathematics Education}. Omaha, NE.
  \item Hetrick, C., Herbst, P., \textbf{Ion, M.}, Brown, A. (Feb. 2023). Building Instructional Capacity Across Difference: Analyzing Transdisciplinary Discourse in a Faculty Learning Community focused on Geometry for Teachers Courses. \textit{25th Annual Conference on Research in Undergraduate Mathematics Education}. Omaha, NE.
  \item \textbf{Ion, M.} (Jul. 2022). Studying Conceptions of the Derivative at Scale: A Machine Learning Approach. \textit{45th Conference of the International Group for the Psychology of Mathematics Education}. Alicante, Spain. 
  \item \textbf{Ion, M.}, Herbst, P. (Feb. 2022). Conceptions of the Derivative: A Natural Language Processing Approach. \textit{Research in Undergraduate Mathematics Education Conference}. Boston, MA. 
  \item Margolis, C., \textbf{Ion, M.}, Herbst, P., Milewski, A., Shultz, M. (Nov. 2020). Understanding instructional capacity for high school geometry as a systemic problem through stakeholder interviews. \textit{Psychology of Mathematics Education, North America}. Mexico. 
  \item Bardelli, E., \textbf{Ion, M.}, Ko, I., Herbst, P. (Apr. 2020). Who Benefits from Mathematics Courses for Teachers? An Analysis of MKT-G Growth During Geometry for Teachers Courses. \textit{American Education Research Association}. San Francisco, CA. 
  \item \textbf{Ion, M.}, Herbst, P., Margolis, C., Milewski, A., Ko, I. (Nov. 2019). Developing Practical Measures To Support the Improvement of Geometry for Teachers Courses. \textit{Psychology of Mathematics Education, North America Annual Conference}. St. Louis, MO. 
  \item Milewski, A., \textbf{Ion, M.}, Herbst, P., Shultz, M., Ko, I., Bleecker, H. (Apr. 2019). Tensions in Teaching Mathematics to Future Teachers: Understanding the Practice of Undergraduate Mathematics Instructors. \textit{American Education Research Association Conference}. Toronto, Canada. 
  \item Herbst, P., Milewski, A., \textbf{Ion, M.}, Bleecker, H. (Oct. 2018). What Influences Do Instructors of the Geometry for Teachers Course Need to Contend With? \textit{Psychology of Mathematics Education, North America}. Greenville, SC. 
\end{list3}
\subsection{\sc Non-peer-reviewed articles and blog posts} 
\begin{list3} 
  \item \textbf{Ion, M.}, Herbst, P. (Nov. 2021). A Contribution to Stewarding the SLOs: Developing SLO Assessment Items and Examining Item Responses. \textit{GeT: The News!, 3}(1). 
  \item Herbst, P., \textbf{Ion, M.} (Nov. 2021). A Deeper Dive into an SLO Item: Examining Students' Ways of Reasoning about Relationships between Euclidean and Non-Euclidean Geometries. \textit{GeT: The News!, 3}(1). 
  \item Boyce, S., \textbf{Ion, M.}, Lai, Y., McLeod, K., Pyzdrowski, L., Sears, R., St. Goar, J. (May 2021). Best-Laid Co-Plans for a Lesson on Creating a Mathematical Definition. \textit{AMS Blogs: On Teaching and Learning Mathematics}.
\end{list3}
\newpage
\section{\sc Presentations}

\subsection{\sc Conference Talks} 
\begin{list3} 
  \item Paulsen, A., Godfrey, J., \textbf{Ion, M.}. (Mar. 2023). Writing Across the Curriculum: a Case Study in Text as Data Methods for Postsecondary Education Policy Research. Denver, CO.
  \item Godfrey, J., Paulson, A., \textbf{Ion, M}. (2023). What Are the Common Contexts for College Writing? \textit{Conference on College Composition and Communication Annual Convention.} Chicago, IL.   
  \item Paulsen, A., \textbf{Ion, M.}, Godfrey, J. (Dec. 2022). Writing Across the Curriculum: a Text as Data Approach. \textit{Causal Inference in Education Research Seminar (CIERS)}. Ann Arbor, MI.
  \item Paulson, A., Bardelli, E., Godfrey, J., \textbf{Ion, M.}, Frisby, M. (Apr. 2022). Who Follows Placement Recommendations? Differential Effects of Non-binding Placement Recommendations on Students' Course-taking Decisions. \textit{American Education Research Association}. San Diego, CA. 
  \item Herbst, P., Stevens, I., Milewski, A., \textbf{Ion, M.}, Ko, I. (Jan. 2020). State of Undergraduate Geometry Courses for Secondary Teachers: Curriculum, Instructional Practices, and Student Achievement. \textit{Joint Mathematics Meeting}. Denver, CO. 
  \item Milewski, A., Herbst, P., \textbf{Ion, M.}, Bleecker, H. (Feb. 2019). Preparing Teachers for Secondary Geometry: Understanding the Tensions in Teaching Undergraduate Mathematics Courses for Future Teachers. \textit{Association of Mathematics Teacher Educators Annual Conference}. Orlando, FL. 
  \item Milewski, A., Herbst, P., Margolis, C., \textbf{Ion, M.}, Ko, I., Akbuga, E. (Jan. 2019). What do we know about courses in Geometry for Secondary Teachers? \textit{Joint Mathematics Meetings}. Baltimore, Maryland. 
\end{list3}

\subsection{\sc Roundtable Discussions}
\begin{list3}
  \item Berzina Pitcher, I., \textbf{Ion, M.}, An, T., Brown, A., Buchbinder, O., Herbst, P., Hetrick, C., Miller, N., Prasad, P., Pyzdrowski, L., St. Goar, J., Sears, R., Szydlik, S., Oshkosh, Vestal, S. (Apr. 2022). Learning and Participating in Scholarship of Teaching and Learning through a Faculty Online Learning Community. \textit{American Education Research Association}. San Diego, CA. 
  \item \textbf{Ion, M.}, Margolis, C. (Mar. 2019). Sources of Justification for College Geometry Instructional Actions. \textit{Graduate Student Community Organization Graduate Student Conference}. Ann Arbor, MI. 
  \item \textbf{Ion, M.} (Mar. 2018). Characterizing University Geometry Courses: An Interview-Based Approach. \textit{Graduate Student Community Organization Graduate Student Conference}. Ann Arbor, MI. 
\end{list3}

\subsection{\sc Posters} 
\begin{list3}
  \item Boyce, B., \textbf{Ion, M.} (Oct. 2023). Geometry Students' Ways of Thinking About Adinkra Symbols. \textit{Psychology of Mathematics Education, North America Annual Conference}. Reno, NV. 
  \item Danai, A., Quimper Osores, A., \textbf{Ion, M.}, Herbst, P. (Apr. 2023). Analysis of Citation Networks of Submitted Manuscripts in Mathematics Education. \textit{Undergraduate Research Opportunity Program (UROP) Symposium}. Ann Arbor, MI. \textit{'Blue Ribbon Outstanding Presenter Award'}
  \item Beckemeyer, R., Brown, A., \textbf{Ion, M.}, Spiteri, A., Herbst, P. (Apr. 2022). How Experience and Knowledge Affect the Breaching Patterns of Secondary Mathematics Teachers. \textit{Undergraduate Research Opportunity Program (UROP) Symposium}. Ann Arbor, MI. \textit{'Blue Ribbon Outstanding Presenter Award'}.
  \item \textbf{Ion, M.}, Bardelli, E., Herbst, P. (Oct. 2018). Learning About the Norms of Teaching Practice: How Can Machine Learning Help Analyze Teachers' Reactions to Scenarios? \textit{Michigan Institute for Data Science Annual Symposium}. Ann Arbor, MI. \textit{Awarded `Most Likely Scientific Impact'}.
\end{list3}

\section{\sc Honors and Awards} 

{\bf University of Michigan}

\begin{tabular}{@{}p{4.0in}p{1.0in}}
Undergraduate Research Opportunity Program (UROP) Mentor Nominee & \multirow{1}{1in}{Feb.~2023}\\
School of Education Travel Grant, School of Education (Pay for travel to international conference) & \multirow{1}{1in}{May~2022}\\
Harold and Vivian Shapiro/John Malik/Jean Forrest Award (\$2000) & \multirow{1}{1in}{Oct.~2021}\\
Jones-Payne-Coxford Award for my scholarly paper, "Measuring Tacit Mathematics Teaching Knowledge: A Natural Language Processing Approach" (One semester of full funding + healthcare) & \multirow{1}{1in}{Mar.~2021}\\
School of Education Scholar Award (Full funding + healthcare for at least four years of study) & \multirow{1}{1in}{Sept.~2017-Present}\\
Most Likely Transformative Science Impact Award for my presentation on "Learning About the Norms of Teaching Practice: How Can Machine Learning Help Analyze Teachers' Reactions to Scenarios" (\$100) & \multirow{1}{1in}{Oct.~2018}\\
\end{tabular}
{\bf California Polytechnic State University}

\begin{tabular}{@{}p{4.0in}p{1.0in}}
Outstanding Teaching Associate Award, (\$500) & \multirow{1}{1in}{Jun.~2015}\\
Marie Porter Lehman Math Educator Scholarship (\$1500) & \multirow{1}{1in}{Jun.~2014}\\
Bryant Russell Memorial Award (\$1500) & \multirow{1}{1in}{Jun.~2013}\\
Volmar A. and Viola I. Folsom Scholarship (\$800) & \multirow{1}{1in}{Jun.~2012}\\
Ralph M. Warten Memorial Scholarship (\$1200) & \multirow{1}{1in}{Jun.~2011}\\
George H. McMeen Scholarships (\$1000) & \multirow{1}{1in}{Jun.~2010}\\

\end{tabular}


\section{\sc Grants and Fellowships} 
  
\begin{tabular}{@{}p{4.0in}p{1.0in}}
  Candidacy Tuition Fellowship, University of Michigan (One semester of full funding + healthcare) & \multirow{1}{1in}{August~2023}\\
  ES Mini Grant, School of Education, University of Michigan (\$1100) & \multirow{1}{1in}{May~2023}\\
  Rackham Debt Management Award, University of Michigan (\$15000) & \multirow{1}{1in}{May~2022}\\
  Educational Studies Summer Grant, University of Michigan (\$2500) & \multirow{1}{1in}{Apr.~2021}\\
  Educational Studies Summer Grant, University of Michigan (\$5000) & \multirow{1}{1in}{Apr.~2019}\\
  Graduate Student Researcher, GeT Support Grant (NSF IUSE Grant \#1725837), University of Michigan (\$2.3 million). P.I.s: Pat Herbst and Amanda Brown. & \multirow{1}{1in}{Sept.~2017-May~2023}\\
\end{tabular}
  

\section{\sc Teaching Experience} 

{\bf \href{https://uplimit.com/}{Uplimit (formerly Corise)}}

\subsection{\sc Teaching Assistant (TA) and Quality Assurance (QA)}
\begin{tabular}{@{}p{4.0in}p{1.0in}}
  \href{https://uplimit.com/course/fine-tuning-language-models}{Fine-tuning Large Language Models (QA)} & \multirow{1}{1in}{Fall 2023}\\
  \href{https://uplimit.com/course/prompt-design-building-ai-products}{Prompt Design and Building AI Products (QA and TA)} & \multirow{1}{1in}{Summer 2023}\\
  \href{https://uplimit.com/course/building-ai-products-with-openai}{Building AI Products with OpenAI (QA and TA)} & \multirow{1}{1in}{Summer 2023}\\
  \href{https://github.com/ramnathv/corise-r-for-ds}{R for Data Science (QA and TA)} & \multirow{1}{1in}{Summer 2023}\\
  Python for Data Science (QA) & Summer 2023\\
\end{tabular}
\begin{tabular}{@{}p{4in}p{1.0in}}

  Uplimit is an online education platform that offers courses in data science, machine learning, and artificial intelligence. \\
  These courses hundreds to thousands of students from all around the world, enrolling upwards of \href{https://www.google.com/maps/d/u/0/viewer?mid=1edk3bLP_d1v1PEF83rqsBxANeaNMGRo&ll=-3.81666561775622e-14%2C12.493146299999921&z=1}{thousands of students}. \\
  My responsibilities include: 
\begin{list1}
  \item Running office hours
  \item Leading project walkthroughs
  \item Replying to student questions about the material in Slack and reviewing and debugging code
  \item A month out from the course starting, I am hired as a quality assurance of the course materials, which includes reviewing and debugging code, weekly meetings with the instructors and Uplimit course management staff, ensuring the course materials are up-to-date, and providing feedback to the course instructors
\end{list1}
 \end{tabular}


{\bf University of Michigan, Ann Arbor, MI}
\subsection{\sc Graduate Student Instructor}
\begin{tabular}{@{}p{4in}p{1.0in}}
Introduction to Quantitative Methods (EDUC 793) & \multirow{2}{1in}{Sept.~2018-Dec.~2022}\\
\end{tabular}
\begin{list1}
    \item Delivering weekly lab instruction on Stata software.
    \item Attending lectures and providing instructional support.
    \item One-on-one office hours with students.
    \item Exam-preparation sessions and creating review materials for the students.
    \item Grading assignments and exams.
\end{list1}
\vspace{.5cm}
{\bf John Hopkins University, Hong Kong \& Seattle}
\subsection{\sc Instructor}

\begin{tabular}{@{}p{4in}p{1.5in}}
    Paradoxes and Infinities & Jul. 2018 \& 2019\\
    \begin{list1}
      \item Curriculum development for "Paradoxes and Infinities".
      \item $100+$ contact hours across 3 weeks, each course had 20 students ages 12-15 from around the world.
      \item Writing evaluations for students.
      \item Supervising a teaching assistant.
  \end{list1}
\end{tabular}


\vspace{.5cm}
{\bf Cal Poly, San Luis Obispo, CA}
\subsection{\sc Lecturer}
\begin{tabular}{@{}p{4in}p{1.0in}}
Calculus for Life Sciences (Math 161) & \multirow{1}{1in}{Summer 2017}\\
Precalculus (Math 118), Calculus for Life Sciences (Math 161) & \multirow{1}{1in}{Spring 2017}\\
Precalculus (Math 118), Trigonometry (Math 119) & \multirow{1}{1in}{Winter 2017}\\
\end{tabular}

\subsection{\sc Graduate Teaching Associate, instructor of record}
\begin{tabular}{@{}p{4in}p{1.0in}}
  Calculus for Business and Economics (Math 221) & \multirow{1}{1in}{Spring 2015}\\
  Precalculus (Math 118), Calculus for Life Sciences (Math 161) & \multirow{1}{1in}{Winter 2015}\\
  Precalculus (Math 118) & \multirow{1}{1in}{Fall 2014}\\
  Calculus for Business and Economics (Math 221) & \multirow{1}{1in}{Spring 2014}\\
  Precalculus (Math 118) & \multirow{1}{1in}{Winter 2014}\\
  Precalculus (Math 116) & \multirow{1}{1in}{Fall 2013}\\
\end{tabular}
\subsection{\sc Calculus Workshop Facilitator}
\begin{tabular}{@{}p{4in}p{1.0in}}
  Calc~I,~II,~III & \multirow{2}{1in}{Sept.~2011-Jun.~2013}\\
  Responsibilities included:\\
\begin{list1}
    \item Attending the content course.
    \item Preparing worksheets, quizzes, and games.
    \item One-on-one student meetings.
    \item Weekly meetings with course instructor and Math Program Staff.
    \item Conducting workshops assisting students with content.
\end{list1}
\end{tabular}

\vspace{.5cm}
{\bf Stanford University, Palo Alto, CA}
\subsection{\sc Residential Counselor/Teaching Assistant} 

\begin{tabular}{@{}p{4in}p{1.0in}}
Stanford Pre-Collegiate Studies Program~  & Jun. 2011 - Aug. 2012\\
\begin{list1}
  \item Provided educational support for gifted middle school students in mathematics courses.
  \item Collected specific instances of good work by individual students to help write evaluations
\end{list1}
\end{tabular}

\section{\sc Mentorship} 
{\bf Graduate Students}

\begin{tabular}{@{}p{4.0in}p{1.0in}}
  Soobin Jeon & \multirow{1}{1in}{2022-2023}\\
  Anna Paulson  & \multirow{1}{1in}{2019-2023}\\
  Jason Godfrey & \multirow{1}{1in}{2018-2023}\\
  Davinia Rodriguez-Wilhelm & \multirow{1}{1in}{2018-2020}\\
  Matt Park & \multirow{1}{1in}{2019-2021}\\
  Scott Bridges & \multirow{1}{1in}{2019-2020}\\
\end{tabular}

{\bf Undergraduate Students}

\begin{tabular}{@{}p{4.0in}p{1.0in}}
  Andre Quimper Osores & \multirow{1}{1in}{2022-2023}\\
  Amirali Danai & \multirow{1}{1in}{2022-2023}\\
  Noah Boudrie & \multirow{1}{1in}{2022-2023}\\
  Robert Beckemeyer & \multirow{1}{1in}{2021-2022}\\
  Andrew Spiteri & \multirow{1}{1in}{2021-2022}\\
  Alan Zhang & \multirow{1}{1in}{2020-2021}\\
  Michael Green & \multirow{1}{1in}{2020-2021}\\
\end{tabular}
\newpage
\section{\sc Professional Memberships \& Affiliations}
\begin{list3}
  \item American Educational Research Association (AERA)
  \item Association of Mathematics Teacher Educators (AMTE)
  \item Graduate Employees' Organization (GEO)
  \item National Council of Teachers of Mathematics (NCTM)
  
\end{list3}
\section{\sc Professional Training}
\begin{tabular}{@{}p{4in}p{1.5in}}
  Natural Language Processing Course, Corise - Comprehensive four-week certification covering the core NLP components such as word vectors, intent classification, and entity recognition using transformer architectures like BERT and GPT. & Feb. 2023 \\
  Statistics and Machine Learning Reading Group - Weekly collaboration focused on applying quantitative research methodologies to social science datasets. Textbooks covered spanned various topics from structural equation modeling to statistical and deep learning. &\multirow{2}{1in}{Sept.~2018-Jun.~2022}\\
  AERA-ICPSR Workshop - One-day session discussing advanced analytic techniques in causal inference. & Feb. 2021 \\
  Deep Learning Workshop - Facilitated by Google. & Nov. 2019\\
  Introduction to Deep Neural Networks with Keras/Tensorflow Workshop - By Greg Teichert. & Jun. 2018 \\
  Big Data Camp - Interdisciplinary team project on NSF grants' success rates based on language use led by the University of Michigan's Interdisciplinary Committee on Organizational Studies. Code available at: \href{https://github.com/mikeion/NSF-Awards-Project}{https://github.com/mikeion/NSF-Awards-Project}. & May. 2018 \\
  Machine Learning for Social Scientists Workshop - By Jake Hofman from Microsoft Research. & Mar. 2018 \\
\end{tabular}



  \section{\sc Service} 
  {\bf United States Peace Corps}\\
  \subsection{\sc Volunteer in Hukuntsi, Botswana}
  \begin{tabular}{@{}p{4in}p{1.0in}}
    Life Skills and Middle School Mathematics Teaching & \multirow{2}{1in}{Jun.~2015-May~2016}
  \end{tabular}
  \begin{list3}
  \item Served as a mentor for an HIV-awareness youth group and a chess club.
  \item Acted as a health promoter while training young people to serve as peer educators, enabling them
  to provide HIV/AIDS education and awareness to other youth and adults in their communities.
  \item Inside and outside the classroom work developing a math curriculum at a low-income junior secondary school.
  \item Advanced-Mid proficiency on the Language Proficiency Interview in the local language (Setswana)
  
  \end{list3}

  {\bf California Men's Colony, San Luis Obispo, CA}
  \subsection{\sc Alternatives to Violence Project, Volunteer}
  \begin{tabular}{@{}p{4in}p{1.0in}}
    Served as a volunteer for a two-day workshop aimed at providing inmates advice on non-violent conflict resolution and strategies for communicating in difficult situations. & \multirow{1}{1in}{December 2014}\\
  \end{tabular}


  \section{\sc Technical Skills}

  \subsection{\sc Programming Languages}
  \begin{list1}
    \item Python
    \item R
    \item Stata
    \item SQL
    \item \LaTeX
    \item M-Plus
    \end{list1}
  
  \subsection{\sc Statistical Models}
  \begin{list1}
  \item Linear and Logistic Regression
  \item Multi-level Models
  \item Psychometric Models
  \item Structural Equation Models
  \item Bayesian Methods
  \end{list1}
  
  \subsection{\sc Machine Learning and Natural Language Processing (ML/NLP)}
  \begin{list1}
  \item Frameworks and Libraries: PyTorch, Transformers, HuggingFace, NLTK, Spacy, Scikit-Learn, Pandas, Numpy, Matplotlib, Seaborn, Plotly, Streamlit, Tensorflow, Keras, Docker
  \item Machine Learning Models: Linear/logistic regression, decision trees, random forests, SVMs, neural networks, CNNs, RNNs, LSTMs, Transformers
  \item Large Language Models (LLMs) and Embeddings: Open-source frameworks/models like Langchain/Langsmith, HuggingFace, LilacML, Streamlit, Gradio, and Closed-source tools (e.g., OpenAI's GPT models). Vector Embeddings tools (e.g., DeepLake, Pinecone, ChromaDB, Faiss, Redis, Qdrant).
  \end{list1}
  
  \subsection{\sc Additional Programming/Software Knowledge}
  \begin{list1}
    \item Git/GitHub
    \item C++
    \item Mathematica
    \item Go
    \item Javascript
    \end{list1}
  \end{resume} 
  \end{document}