% !TEX program = lualatex
% !TEX enableSynctex = true

\documentclass[a4paper,11pt]{article}

\usepackage         {fontspec}    % Font selection
\usepackage         {fontawesome5} % Symbols/icons
\usepackage         {xspace}      % Better spacing

% Fonts - using more widely available alternatives that match the style
\defaultfontfeatures{Ligatures=TeX}
\setmainfont[Numbers={Proportional,OldStyle}]{TeX Gyre Termes}
\setsansfont[Scale=MatchLowercase]{TeX Gyre Heros}
\setmonofont[Scale=MatchLowercase]{TeX Gyre Cursor}

% Document layout
\usepackage{geometry}
\geometry{
  a4paper,
  textwidth      = 14.0cm,
  textheight     = 25.0cm,
  marginparwidth =  2.5cm
}

\setlength{\parindent}{0pt}     % No indentations for paragraphs
\setlength{\parskip}{0.5em}     % Space between paragraphs
\setlength{\skip\footins}{2cm}  % Reduced footer distance

% Margin notes for dates
\usepackage{marginnote}
\renewcommand*{\raggedleftmarginnote}{}
\newcommand{\years}[1]{%
  {\reversemarginpar\strut\marginnote{{\small#1}}}%
}

% Combined years and item command for enumerated lists
\newcommand{\yearsitem}[1]{%
  \item {\reversemarginpar\strut\marginnote{{\small#1}}}%
}

% Equal contribution for authors
\newcommand{\authorequal}{\kern-0.1em\textsuperscript{\dagger}}

% Section styling
\usepackage{sectsty}
\allsectionsfont{\mdseries\scshape}

% Typographical adjustments
\usepackage{microtype}

% Colors
\usepackage{xcolor}
\definecolor{cardinal}{RGB}{196, 30, 58}

% PDF setup
\usepackage[
  bookmarks,  
  colorlinks, 
  breaklinks  
]{hyperref}

\hypersetup{%
  urlcolor=cardinal,
  linkcolor=cardinal,
  pdfauthor={Michael Ion},
  pdftitle={Michael Ion: Curriculum Vitae}
}

% Reject orphaned and widowed lines
\clubpenalty         = 10000
\displaywidowpenalty = 10000
\widowpenalty        = 10000
\usepackage[defaultlines=100,all]{nowidow}

% Add these new settings for itemize
\usepackage{enumitem}
\setlist[itemize]{leftmargin=*, nosep, topsep=0pt, partopsep=0pt, parsep=0pt}

\begin{document}

{
  \begin{minipage}[t]{0.7\textwidth}
    \Huge \textsc{Michael (Mike) Ion}
    \vspace{0.5em}
    
    \normalsize Postdoctoral Research Fellow\\
    School of Information\\
    University of Michigan
  \end{minipage}
  \hfill
  \begin{minipage}[t]{0.3\textwidth}
    \raggedleft
    \small
    \faEnvelope~\href{mailto:mikeion@umich.edu}{mikeion@umich.edu}\\
    \faGlobe~\href{https://mikeion.com}{mikeion.com}\\
    \faGithub~\href{https://github.com/mikeion}{github.com/mikeion}\\[1em]
    \textit{Last updated: March 2025}
  \end{minipage}
}

\vspace{1em}

I'm an educator, researcher, and developer working at the intersection of mathematics, statistics, data science, and AI. My work integrates computational methods and statistical modeling with insights from educational research, exploring how teachers and learners strategically interact to build effective learning environments. I am particularly committed to developing innovative tools and strategies that account for learners' diverse backgrounds, support adaptive teaching methods, and promote meaningful feedback and learner agency.

\section*{Academic Positions}

\years{2024--Present}\textbf{Postdoctoral Research Fellow}, School of Information\\
University of Michigan, Ann Arbor\\
Research Advisor: \href{https://websites.umich.edu/~kevynct/}{Kevyn Collins-Thompson}

\section*{Education}

\years{2017--2024} \textbf{Ph.D.} in Mathematics Education at University of Michigan, Ann Arbor\\
\emph{Thesis:} \emph{Beyond the Classroom: Exploring Mathematics Engagement in Online Communities with Natural Language Processing}\\
\emph{Advisor:} \href{https://deborahloewenbergball.com/}{Deborah Ball}, \emph{Committee:} (Cognate) \href{https://jurgens.people.si.umich.edu/}{David Jurgens}, \href{https://marsal.umich.edu/directory/faculty-staff/christopher-quintana}{Christopher Quintana}, \href{https://www.gse.harvard.edu/directory/faculty/ying-xu}{Ying Xu}

\years{2013--2015} \textbf{M.S.} in Mathematics at California Polytechnic State University, San Luis Obispo

\years{2009--2013} \textbf{B.S.} in Mathematics at California Polytechnic State University, San Luis Obispo

\section*{Publications}

\begin{enumerate}

\subsection*{Journal Publications and Peer-Reviewed Conference Proceedings}

\yearsitem{In review}\textbf{Ion, M.}, Collins-Thompson, K. (2025). Bayesian Hierarchical Modeling of Large-Scale Math Tutoring Dialogues. \emph{Joint Statistical Meetings}. (In review)

\yearsitem{In preparation}\textbf{Ion, M.}, Collins-Thompson, K., Asthana, S. (In preparation). Simulated Teaching and Learning at Scale: Balancing Fidelity and Effectiveness in Tutoring Interactions. \emph{Learning @ Scale 2025}. 

\item \textbf{Ion, M.}, Ball, Lowenberg D. (In preparation). Teaching and Learning in the Age of Generative AI: Understanding the Human Work of Instruction. \emph{For the Learning of Mathematics}

\yearsitem{2023}Herbst, P., Brown, A.M., \textbf{Ion, M.}, Margolis, C. (2023). Teaching Geometry for Secondary Teachers: What are the Tensions Instructors Need to Manage? \emph{International Journal of Research in Undergraduate Mathematics Education}. \href{https://doi-org.proxy.lib.umich.edu/10.1007/s40753-023-00216-0}{doi: 10.1007/s40753-023-00216-0}

\item Gere, A., Godfrey, J., Griffin, M., \textbf{Ion, M.}, Limlamai, N., Moos, A., Van Zanen, K. (2023). Alumni Perspectives on General Education: How Writing Can Increase What We Know. \emph{Journal of General Education, 70}(1-2), 149-175. \href{https://doi.org/10.5325/jgeneeduc.70.1-2.0149}{doi: 10.5325/jgeneeduc.70.1-2.0149}

\item \textbf{Ion, M.}, Herbst, P., Ko, I., Hetrick, C. (2023). Agreeing on objectives of geometry for teachers’ courses: Feedback from instructors on an initial list. \emph{Psychology of Mathematics Education, North America Annual Conference}. Reno, NV.

\item Brown, A., Herbst, P., \textbf{Ion, M.} (2023). How Instructors of Undergraduate Mathematics Courses Manage Tensions Related to Teaching Courses for Teachers. \emph{Psychology of Mathematics Education, North America Annual Conference}. Reno, NV.

\item Boyce, S., An, T., Pyzdrowski, L., Oppong-Wadie, K., \textbf{Ion, M.}, St. Goar, J. (2023). Learning from Lesson Study in the College Geometry Classroom. \emph{25th Annual Conference on Research in Undergraduate Mathematics Education}. Omaha, NE.

\item Hetrick, C., Herbst, P., \textbf{Ion, M.}, Brown, A. (2023). Building Instructional Capacity Across Difference: Analyzing Transdisciplinary Discourse in a Faculty Learning Community focused on Geometry for Teachers Courses. \emph{25th Annual Conference on Research in Undergraduate Mathematics Education}. Omaha, NE.

\item Hetrick, C., Herbst, P.G., Brown, A.M., \textbf{Ion, M.} (2023). Contention and Coalescence in Mathematical Knowledge: Undergraduate Geometry Instructors' Cooperative Design of Student Learning Objectives. \emph{American Educational Research Association}. San Diego, CA.

\item \textbf{Ion, M.}, Herbst, P. (2022). Conceptions of the Derivative: A Natural Language Processing Approach. \emph{Research in Undergraduate Mathematics Education Conference}. Boston, MA.

\yearsitem{2020}Margolis, C., \textbf{Ion, M.}, Herbst, P., Milewski, A., Shultz, M. (2020). Understanding instructional capacity for high school geometry as a systemic problem through stakeholder interviews. \emph{Psychology of Mathematics Education, North America}. Mexico.

\item Bardelli, E., \textbf{Ion, M.}, Ko, I., Herbst, P. (2020). Who Benefits from Mathematics Courses for Teachers? An Analysis of MKT-G Growth During Geometry for Teachers Courses. \emph{American Education Research Association}. San Francisco, CA.

\yearsitem{2019}\textbf{Ion, M.}, Herbst, P., Margolis, C., Milewski, A., Ko, I. (2019). Developing Practical Measures To Support the Improvement of Geometry for Teachers Courses. \emph{Psychology of Mathematics Education, North America Annual Conference}. St. Louis, MO.

\item Milewski, A., \textbf{Ion, M.}, Herbst, P., Shultz, M., Ko, I., Bleecker, H. (2019). Tensions in Teaching Mathematics to Future Teachers: Understanding the Practice of Undergraduate Mathematics Instructors. \emph{American Education Research Association Conference}. Toronto, Canada.

\yearsitem{2018}Herbst, P., Milewski, A., \textbf{Ion, M.}, Bleecker, H. (2018). What Influences Do Instructors of the Geometry for Teachers Course Need to Contend With? \emph{Psychology of Mathematics Education, North America}. Greenville, SC.



\subsection*{Book Chapters}

\yearsitem{2024}An, T., Boyce, S., Brown, A., Buchbinder, O., Cohen, S., Dumitrascu, D., Escuadro, H., Herbst, P., \textbf{Ion, M.}, Krupa, E., Miller, N., Pyzdrowski, L., Sears, R., St. Goar, J., Szydlik, S., Vestal, S. (2024). (Toward) Essential student learning objectives for teaching geometry to pre-service secondary teachers. \emph{The AMTE Handbook of Mathematics Teacher Education: Reflection on Past, Present and Future – Paving the Way for the Future of Mathematics Teacher Education}, 175-197.

\subsection*{Non-peer-reviewed Articles}

\yearsitem{2021}\textbf{Ion, M.}, Herbst, P. (2021). A Contribution to Stewarding the SLOs: Developing SLO Assessment Items and Examining Item Responses. \emph{GeT: The News!, 3}(1).

\item Herbst, P., \textbf{Ion, M.} (2021). A Deeper Dive into an SLO Item: Examining Students' Ways of Reasoning about Relationships between Euclidean and Non-Euclidean Geometries. \emph{GeT: The News!, 3}(1).

\item Boyce, S., \textbf{Ion, M.}, Lai, Y., McLeod, K., Pyzdrowski, L., Sears, R., St. Goar, J. (2021). Best-Laid Co-Plans for a Lesson on Creating a Mathematical Definition. \emph{AMS Blogs: On Teaching and Learning Mathematics}.

\subsection*{Posters}

\yearsitem{2025}\textbf{Ion, M.}, Asthana, S., Jiao, F., Wang, T., Collins-Thompson, K. (2025). Adaptive Knowledge Assessment in Simulated Coding Interviews. \emph{iRAISE Workshop at AAAI Conference}. Philadelphia, PA.

\yearsitem{2023}Boyce, B., \textbf{Ion, M.} (2023). Geometry Students' Ways of Thinking About Adinkra Symbols. \emph{Psychology of Mathematics Education, North America Annual Conference}. Reno, NV.

\item Danai, A., Quimper Osores, A., \textbf{Ion, M.}, Herbst, P. (2023). Analysis of Citation Networks of Submitted Manuscripts in Mathematics Education. \emph{Undergraduate Research Opportunity Program (UROP) Symposium}. Ann Arbor, MI. \emph{'Blue Ribbon Outstanding Presenter Award'}

\yearsitem{2022}Beckemeyer, R., Brown, A., \textbf{Ion, M.}, Spiteri, A., Herbst, P. (2022). How Experience and Knowledge Affect the Breaching Patterns of Secondary Mathematics Teachers. \emph{Undergraduate Research Opportunity Program (UROP) Symposium}. Ann Arbor, MI. \emph{'Blue Ribbon Outstanding Presenter Award'}.

\item \textbf{Ion, M.} (2022). Studying Conceptions of the Derivative at Scale: A Machine Learning Approach. \emph{45th Conference of the International Group for the Psychology of Mathematics Education}. Alicante, Spain.

\item Berzina Pitcher, I., \textbf{Ion, M.}, An, T., Brown, A., Buchbinder, O., Herbst, P., Hetrick, C., Miller, N., Prasad, P., Pyzdrowski, L., St. Goar, J., Sears, R., Szydlik, S., Oshkosh, Vestal, S. (2022). Learning and Participating in Scholarship of Teaching and Learning through a Faculty Online Learning Community. \emph{American Education Research Association}. San Diego, CA.

\yearsitem{2021}Herbst, P. G., Milewski, A. M., \textbf{Ion, M.}, Ko, I. (2021). Preparing Teachers for Secondary Geometry: Helping Shape the Geometry Course for Teachers. \emph{National Council of Teachers of Mathematics}. Virtual.



\yearsitem{2020}Herbst, P., Stevens, I., Milewski, A., \textbf{Ion, M.}, Ko, I. (2020). State of Undergraduate Geometry Courses for Secondary Teachers: Curriculum, Instructional Practices, and Student Achievement. \emph{Joint Mathematics Meeting}. Denver, CO.

\yearsitem{2019}Milewski, A., Herbst, P., \textbf{Ion, M.}, Bleecker, H. (2019). Preparing Teachers for Secondary Geometry: Understanding the Tensions in Teaching Undergraduate Mathematics Courses for Future Teachers. \emph{Association of Mathematics Teacher Educators Annual Conference}. Orlando, FL.

\yearsitem{2018}\textbf{Ion, M.}, Bardelli, E., Herbst, P. (2018). Learning About the Norms of Teaching Practice: How Can Machine Learning Help Analyze Teachers' Reactions to Scenarios? \emph{Michigan Institute for Data Science Annual Symposium}. Ann Arbor, MI. \emph{Awarded 'Most Likely Scientific Impact'}.

\end{enumerate}

\section*{Research Grants}

\begin{enumerate}
\subsection*{In Review}

\yearsitem{2025}\textbf{Senior Personnel \& Co-Author}, Instructor-centered Holistic Modeling of Student Engagement and Progress in Data Science, submitted to NSF 23-624: Research on Innovative Technologies for Enhanced Learning (RITEL)
\begin{itemize}
    \item With K. Collins-Thompson (PI), C. Brooks (co-PI), and S. Oney (co-PI)
    \item Total amount requested: \$750,000
\end{itemize}

\item \textbf{Senior Personnel and Co-Author}, Test Beds for Higher Education, submitted to NSF 24-111: Planning Grants to Create Artificial Intelligence (AI)-Ready Test Beds
\begin{itemize}
    \item With K. Collins-Thompson (PI) and C. Brooks (co-PI)
    \item Total amount requested: \$100,000
\end{itemize}

\subsection*{Awarded}

\yearsitem{2025}\textbf{Co-Principal Investigator}, Learning Through Technical Interviews: Combining Data Science Mentorship with AI-Powered Practice (Academic Innovation Fund) (\$12,435)
\begin{itemize}
    \item With K. Collins-Thompson (co-PI)
\end{itemize}

\years{2017--2023}\textbf{Graduate Research Assistant}, GeT Support: An online professional learning community to support the geometry course for teachers (NSF IUSE Grant \#1725837) (\$2.3 million)
\begin{itemize}
    \item PI: P. Herbst
\end{itemize}
\end{enumerate}


\section*{Invited talks \& guest lectures}

\begin{enumerate}
\yearsitem{2025}\textbf{Ion, M.} (2025). Text-as-Data in Mathematics Education: Harnessing LLMs to Analyze Student Conversations at Scale. \emph{AMS Special Session on SoTL: Connecting Generative AI and Scholarly Inquiry to Improve Teaching and Learning, Joint Mathematics Meeting (JMM)}. Seattle, WA.

\yearsitem{2024}\textbf{Ion, M.} (2024). Use of LLMs and Langchain to Extract Insights about Mathematics Conversations at Scale. A 45-minute talk given to a master's level University of Michigan data science course, SIADS 676: Applications of Generative AI.

\yearsitem{2023}\textbf{Ion, M.} (2023). New Directions in Education Research: Harnessing Text-as-Data Methods. San Diego State University, CA. On-Campus Job Talk for Tenure-Track Statistics Education Professor Position.

\item Paulson, A., Godfrey, J., \textbf{Ion, M.} (2023). Writing Across the Curriculum: a Case Study in Text as Data Methods for Postsecondary Education Policy Research. Denver, CO.

\item Godfrey, J., Paulson, A., \textbf{Ion, M.} (2023). What Are the Common Contexts for College Writing? \emph{Conference on College Composition and Communication Annual Convention}. Chicago, IL.

\yearsitem{2022}Paulson, A., \textbf{Ion, M.}, Godfrey, J. (2022). Writing Across the Curriculum: a Text as Data Approach. \emph{Causal Inference in Education Research Seminar (CIERS)}. Ann Arbor, MI.

\item Paulson, A., Bardelli, E., Godfrey, J., \textbf{Ion, M.}, Frisby, M. (2022). Who Follows Placement Recommendations? Differential Effects of Non-binding Placement Recommendations on Students' Course-taking Decisions. \emph{American Education Research Association}. San Diego, CA.

\item Berzina Pitcher, I., \textbf{Ion, M.}, An, T., Brown, A., Buchbinder, O., Herbst, P., Hetrick, C., Miller, N., Prasad, P., Pyzdrowski, L., St. Goar, J., Sears, R., Szydlik, S., Oshkosh, Vestal, S. (2022). Learning and Participating in Scholarship of Teaching and Learning through a Faculty Online Learning Community. \emph{American Education Research Association}. San Diego, CA.

\yearsitem{2021}Herbst, P. G., Milewski, A. M., \textbf{Ion, M.}, Ko, I. (2021). Preparing Teachers for Secondary Geometry: Helping Shape the Geometry Course for Teachers. \emph{National Council of Teachers of Mathematics}. Virtual.

\yearsitem{2020}Herbst, P., Stevens, I., Milewski, A., \textbf{Ion, M.}, Ko, I. (2020). State of Undergraduate Geometry Courses for Secondary Teachers: Curriculum, Instructional Practices, and Student Achievement. \emph{Joint Mathematics Meeting}. Denver, CO.

\yearsitem{2019}\textbf{Ion, M.}, Margolis, C. (2019). Sources of Justification for College Geometry Instructional Actions. \emph{Graduate Student Community Organization Graduate Student Conference}. Ann Arbor, MI.

\item Milewski, A., Herbst, P., \textbf{Ion, M.}, Bleecker, H. (2019). What do we know about courses in Geometry for Secondary Teachers? \emph{Joint Mathematics Meetings}. Baltimore, Maryland.

\yearsitem{2018}\textbf{Ion, M.} (2018). Characterizing University Geometry Courses: An Interview-Based Approach. \emph{Graduate Student Community Organization Graduate Student Conference}. Ann Arbor, MI.

\end{enumerate}
\section*{Teaching}

\subsection*{Uplimit (formerly Corise)}

\years{2023} \textbf{Teaching Assistant and Quality Assurance}
\begin{itemize}
    \item Fine-tuning Large Language Models
    \item Prompt Design and Building AI Products
    \item Building AI Products with OpenAI
    \item R for Data Science
\end{itemize}
\years{2023} \textbf{Teaching Assistant}, Python for Data Science

\subsection*{University of Michigan}

\years{2018--2019} \textbf{Graduate Student Instructor}, Introduction to Quantitative Methods (EDUC 793)
\begin{itemize}
    \item Fall 2019: 1 section
    \item Fall 2018: 1 section
\end{itemize}

\subsection*{Johns Hopkins University}

\years{2018--2019} \textbf{Lead Instructor}, Paradoxes and Infinities
\begin{itemize}
    \item Summer 2019: Hong Kong program
    \item Summer 2018: Seattle program
\end{itemize}

\subsection*{California Polytechnic State University}

\years{2017} \textbf{Instructor of Record}
\begin{itemize}
    \item Fall 2017: Calculus for Life Sciences (Math 161), 2 sections
    \item Spring 2017: Precalculus (Math 118), 2 sections
    \item Spring 2017: Trigonometry (Math 119), 1 section
\end{itemize}

\years{2015} \textbf{Instructor of Record}
\begin{itemize}
    \item Spring 2015: Calculus for Business and Economics (Math 221), 2 sections
\end{itemize}

\years{2014--2015} \textbf{Instructor of Record}
\begin{itemize}
    \item Winter 2015: Precalculus (Math 118), 2 sections
    \item Fall 2014: Precalculus (Math 118), 2 sections
\end{itemize}

\years{2013--2014} \textbf{Instructor of Record}
\begin{itemize}
    \item Spring 2015: Calculus for Business and Economics (Math 221), 2 sections
    \item Winter 2014: Precalculus (Math 118), 1 section
    \item Fall 2013: Precalculus I (Math 116), 1 section
\end{itemize}

\years{2011--2013} \textbf{Workshop Facilitator}, Calculus Workshop
\begin{itemize}
    \item Spring 2013: Calculus II Workshop, 1 section
    \item Winter 2013: Calculus III Workshop, 1 section
    \item Fall 2012: Calculus II Workshop, 1 section
    \item Spring 2012: Calculus III Workshop, 1 section
    \item Winter 2012: Calculus II Workshop, 1 section
    \item Fall 2011: Calculus I Workshop, 1 section
\end{itemize}

\section*{Students}

\subsection*{Graduate Students}

\years{2024--Present} Megan Pouncy

\years{2022--2023} Soobin Jeon

\years{2019--2023} Anna Paulson

\years{2018--2023} Jason Godfrey

\years{2019--2021} Matt Park

\years{2018--2020} Davinia Rodriguez-Wilhelm

\years{2019--2020} Scott Bridges

\subsection*{Undergraduate Students}

\years{2024--Present} Fengquan Jiao, Tianyi Wang

\years{2022--2023} Andre Quimper Osores, Amirali Danai, Noah Boudrie

\years{2021--2022} Robert Beckemeyer, Andrew Spiteri

\years{2020--2021} Alan Zhang, Michael Green

\section*{Service to the community}

\subsection*{Editorial Board Memberships}

\years{2022--2023} Editorial Assistant, Journal for Research in Mathematics Education (JRME)

\subsection*{Reviews}

\years{2022--2025} \textbf{Journals}
\begin{itemize}
    \item Investigations in Mathematics Learning
    \item Journal of Engineering Education
    \item Journal for Research in Mathematics Education
\end{itemize}

\years{2017--2024} \textbf{Conferences}
\begin{itemize}
    \item Psychology of Mathematics Education (PME-NA)
    \item Research in Undergraduate Mathematics Education (RUME)
    \item Geometry Teacher (GeT) Support Conference
    \item University of Michigan Graduate Student Conference (GSCO)
\end{itemize}

\subsection*{Professional Memberships}

\years{2024--Present} American Statistical Association (ASA)

\years{2017--2024} American Educational Research Association (AERA)

\years{2017--2020} Association of Mathematics Teacher Educators (AMTE)

\years{2017--2024} Graduate Employees' Organization (GEO)

\subsection*{Outreach Activities}

\years{2015--2016} \textbf{United States Peace Corps Volunteer}, Hukuntsi, Botswana

\years{2014} \textbf{Alternatives to Violence Project}, California Men's Colony, San Luis Obispo, CA

\subsection*{Professional Development Provided}

\years{2018} Herbst, P., Milewski, A., Boileau, N., \textbf{Ion, M.} (2018). Integrating Geogebra into HS Geometry. 3-Day Workshop in Ann Arbor Public Schools.

\section*{Professional Training}


\years{2018--2022} Statistics and Machine Learning Reading Group

\years{2021} AERA-ICPSR Workshop

\years{2019} Deep Learning Workshop - Facilitated by Google

\years{2018} Introduction to Deep Neural Networks with Keras/Tensorflow Workshop

\years{2018} Big Data Camp

\years{2018} Machine Learning for Social Scientists Workshop

\section*{Technical Skills}

\years{} \textbf{Programming Languages}: Python, Javascript, Next.js, R, Stata, SQL, \LaTeX, M-Plus, C++

\years{} \textbf{Statistical Models}: Linear and Logistic Regression, Multi-level Models, Psychometric Models, Structural Equation Models, Bayesian Methods, Causal Inference Methods, Time Series Analysis

\years{} \textbf{Machine Learning and NLP}: PyTorch, Transformers, HuggingFace, NLTK, Spacy, Scikit-Learn, Pandas, Numpy, Matplotlib, Tensorflow, LangChain, Vector Databases (Pinecone, ChromaDB, Faiss)

\section*{Research Experience}

\years{2017--2023} \textbf{Research Assistant}, GRIP Lab, University of Michigan

\years{2020--2022} \textbf{Research Assistant}, College and Beyond II Project (Mellon Grant)

\years{2019} \textbf{Research Assistant}, Wolverine Pathways Curriculum Development Project

\years{2013} \textbf{Research Experience for Undergraduates (REU)}, California Polytechnic State University

\section*{Honors, Awards \& Fellowships}

\years{2023} Candidacy Tuition Fellowship, University of Michigan (One semester funding + healthcare)

\years{2023} ES Mini Grant, School of Education, University of Michigan (\$1100)

\years{2023} Undergraduate Research Opportunity Program (UROP) Mentor Nominee

\years{2022} Rackham Debt Management Award, University of Michigan (\$15000)

\years{2022} School of Education Travel Grant

\years{2021} Harold and Vivian Shapiro/John Malik/Jean Forrest Award (\$2000)

\years{2021} Jones-Payne-Coxford Award (One semester of full funding + healthcare)

\years{2021} Educational Studies Summer Grant, University of Michigan (\$2500)

\years{2019} Educational Studies Summer Grant, University of Michigan (\$5000)

\years{2018} Most Likely Transformative Science Impact Award (\$100)

\years{2017--Present} School of Education Scholar Award (Full funding + healthcare for at least four years of study)

\years{2015} Outstanding Teaching Associate Award (\$500), California Polytechnic State University

\years{2014} Marie Porter Lehman Math Educator Scholarship (\$1500), California Polytechnic State University

\years{2013} Bryant Russell Memorial Award (\$1500), California Polytechnic State University

\end{document}